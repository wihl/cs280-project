%%
%% Copyright 2007, 2008, 2009 Elsevier Ltd
%%
%% This file is part of the 'Elsarticle Bundle'.
%% ---------------------------------------------
%%
%% It may be distributed under the conditions of the LaTeX Project Public
%% License, either version 1.2 of this license or (at your option) any
%% later version.  The latest version of this license is in
%%    http://www.latex-project.org/lppl.txt
%% and version 1.2 or later is part of all distributions of LaTeX
%% version 1999/12/01 or later.
%%
%% The list of all files belonging to the 'Elsarticle Bundle' is
%% given in the file `manifest.txt'.
%%

%% Template article for Elsevier's document class `elsarticle'
%% with harvard style bibliographic references
%% SP 2008/03/01
%%
%%
%%
%% $Id: elsarticle-template-harv.tex 4 2009-10-24 08:22:58Z rishi $
%%
%%
\documentclass[final,authoryear,11pt,times]{elsarticle}

%% Use the option review to obtain double line spacing
%% \documentclass[authoryear,preprint,review,12pt]{elsarticle}

%% Use the options 1p,twocolumn; 3p; 3p,twocolumn; 5p; or 5p,twocolumn
%% for a journal layout:
%% \documentclass[final,authoryear,1p,times]{elsarticle}
%% \documentclass[final,authoryear,1p,times,twocolumn]{elsarticle}
%% \documentclass[final,authoryear,3p,times]{elsarticle}
%% \documentclass[final,authoryear,3p,times,twocolumn]{elsarticle}

%% \documentclass[final,authoryear,5p,times,twocolumn]{elsarticle}

%% if you use PostScript figures in your article
%% use the graphics package for simple commands
%% \usepackage{graphics}
%% or use the graphicx package for more complicated commands
%% \usepackage{graphicx}
%% or use the epsfig package if you prefer to use the old commands
%% \usepackage{epsfig}

%% The amssymb package provides various useful mathematical symbols
\usepackage{amssymb}

\usepackage[margin=1.25in]{geometry}


\usepackage{setspace}
\onehalfspacing

%% The amsthm package provides extended theorem environments
%% \usepackage{amsthm}

%% The lineno packages adds line numbers. Start line numbering with
%% \begin{linenumbers}, end it with \end{linenumbers}. Or switch it on
%% for the whole article with \linenumbers after \end{frontmatter}.
%% \usepackage{lineno}

%% natbib.sty is loaded by default. However, natbib options can be
%% provided with \biboptions{...} command. Following options are
%% valid:

%%   round  -  round parentheses are used (default)
%%   square -  square brackets are used   [option]
%%   curly  -  curly braces are used      {option}
%%   angle  -  angle brackets are used    <option>
%%   semicolon  -  multiple citations separated by semi-colon (default)
%%   colon  - same as semicolon, an earlier confusion
%%   comma  -  separated by comma
%%   authoryear - selects author-year citations (default)
%%   numbers-  selects numerical citations
%%   super  -  numerical citations as superscripts
%%   sort   -  sorts multiple citations according to order in ref. list
%%   sort&compress   -  like sort, but also compresses numerical citations
%%   compress - compresses without sorting
%%   longnamesfirst  -  makes first citation full author list
%%
%% \biboptions{longnamesfirst,comma}

% \biboptions{}

\journal{cs280r - Final Project Report}

\begin{document}

\begin{frontmatter}

%% Title, authors and addresses

%% use the tnoteref command within \title for footnotes;
%% use the tnotetext command for the associated footnote;
%% use the fnref command within \author or \address for footnotes;
%% use the fntext command for the associated footnote;
%% use the corref command within \author for corresponding author footnotes;
%% use the cortext command for the associated footnote;
%% use the ead command for the email address,
%% and the form \ead[url] for the home page:
%%
%% \title{Title\tnoteref{label1}}
%% \tnotetext[label1]{}
%% \author{Name\corref{cor1}\fnref{label2}}
%% \ead{email address}
%% \ead[url]{home page}
%% \fntext[label2]{}
%% \cortext[cor1]{}
%% \address{Address\fnref{label3}}
%% \fntext[label3]{}

\title{CS280r Final Project Report \\ $A \rho \mu o \nu \acute{\iota} \alpha$ (Harmonia): A System for Collaborative Music Composition}

%% use optional labels to link authors explicitly to addresses:
%% \author[label1,label2]{<author name>}
%% \address[label1]{<address>}
%% \address[label2]{<address>}

\author{Mark Goldstein, David Wihl}
\address{\{markgoldstein,davidwihl\}@g.harvard.edu}

\begin{abstract}
%% Text of abstract

Increasing productivity of music composition has many positive benefits. Listeners
would appreciate individually tailored music to their emotional needs and context.
Composers would be facilitated by greater and more diverse cooperation yielding more
innovative music. Composition agents could assist in the generation of repetitive or 
experimental musical forms. Therapists can use music as part of a treatment plan 
for autism and
many other disorders. The system we propose attempts to address these myriad needs 
by offering two key innovations: a SharedPlan with collaborative versioning to mediate the workflow of a composition, an algorithmic evaluation of a composition against the intention of the
SharedPlan to provide guidance to both human and agent composers.

\end{abstract}
\end{frontmatter}

% \linenumbers

%% main text
\section{Introduction}
\label{sec:introduction}
TODO  Should contain an overview of the problem to be addressed, the approach taken to   
  address that problem, and the results of that approach. Should provide the reader with a  
  road map for how your argument will be developed in the other sections of the paper.


RESEARCH TODO
\begin{itemize}
\item Mark: talk to composers and incorporate their feedback and UI suggestions
\item David: speak to David Greenberg to incorporate feedback and UI
\end{itemize}

%
%\section{Body of the Paper}
%
%\begin{itemize}
%	\item {\bf  Experimental Design.} A description of the experiment that was run; enough detail should be provided   
%that the reader could reasonably duplicate the experiment. Results should not be 
%reported in this section.
%	\item {\bf Results.} A report of the results of the experiments, and their significance.
%\end{itemize}


%\subsection{Citations}

%Here are two examples of how to cite a paper properly:
%\begin{itemize}
%	\item \citet{bernstein2000complexity} shows that ... 
%	\item Prior work has shown that ... \citep{bernstein2000complexity}.
%\end{itemize}

%%  \citet{key}  ==>>  Jones et al. (1990)
%%  \citep{key}  ==>>  (Jones et al., 1990)


\section{Related Work}
TODO Discussion of previous important, similar work in the area with comparison to the particular approach taken and results of the paper. Avoid simply providing a laundry list of other work that is somehow related to the subject of the paper. This section should contain brief, in depth discussions of the work most similar to your project, i.e., to research that takes an approach to the problem or produces results with which your project should be compared. As is always the case with written work, throughout the paper you should have citations to work that you draw on. For example, if you have adapted a system, include a citation to the system when you first mention it; if you are extending a formalization, include a citation to the original on first mention. If you are unclear about whether a simple citation suffices or an extended discussion is needed in the Related Work section, look at the papers read for class this semester for models. If you are still unsure, check with the teaching staff.

\section{Workflow Overview}

git + intention + algorithmic eval

incorporate SharedPlans

Current git / music solutions

\section{Algorithmic Evaluator}

Current design

MIDI

entropy discussion

KL Divergence

\section{Use Cases}

\subsection{Individual User, Individual Composer}

\subsection{Multiple Composers}

TODO: include failure modes

\subsection{Therapist with Agent - Human Composition Team}

high volume necessity


\section{Discussion}

\subsection{Enhancing or Stifling Creativity}

Notes: evaluation is optional. Can be ignored by committer.

\subsection{Limitations}

Collaboration is offline, not real-time

Current music representation is discrete MIDI, not audio. Limits for vocals, ocean sounds

Presume that reliable corpus-based genre and mood classification solutions exist, particularly information retrieval procedures



\section{Conclusion}
 Two Novel Contributions:
 \begin{itemize}
\item Collaborative music composition system 
Intentionality, SharedPlan and Agents
\item Algorithmic evaluation of composition against intention
\end{itemize}

\section {Future work}
\begin{itemize}
\item Improved agent composition
\item Intelligent ad hoc composition
\item Facilitator of scalable music composition
\item improved evaluator, possibly RNN based
\end{itemize}


%% The Appendices part is started with the command \appendix;
%% appendix sections are then done as normal sections
%% \appendix

%% \section{}
%% \label{}

%% References
%%
%% Following citation commands can be used in the body text:
%%
%%  \citet{key}  ==>>  Jones et al. (1990)
%%  \citep{key}  ==>>  (Jones et al., 1990)
%%
%% Multiple citations as normal:
%% \citep{key1,key2}         ==>> (Jones et al., 1990; Smith, 1989)
%%                            or  (Jones et al., 1990, 1991)
%%                            or  (Jones et al., 1990a,b)
%% \cite{key} is the equivalent of \citet{key} in author-year mode
%%
%% Full author lists may be forced with \citet* or \citep*, e.g.
%%   \citep*{key}            ==>> (Jones, Baker, and Williams, 1990)
%%
%% Optional notes as:
%%   \citep[chap. 2]{key}    ==>> (Jones et al., 1990, chap. 2)
%%   \citep[e.g.,][]{key}    ==>> (e.g., Jones et al., 1990)
%%   \citep[see][pg. 34]{key}==>> (see Jones et al., 1990, pg. 34)
%%  (Note: in standard LaTeX, only one note is allowed, after the ref.
%%   Here, one note is like the standard, two make pre- and post-notes.)
%%
%%   \citealt{key}          ==>> Jones et al. 1990
%%   \citealt*{key}         ==>> Jones, Baker, and Williams 1990
%%   \citealp{key}          ==>> Jones et al., 1990
%%   \citealp*{key}         ==>> Jones, Baker, and Williams, 1990
%%
%% Additional citation possibilities
%%   \citeauthor{key}       ==>> Jones et al.
%%   \citeauthor*{key}      ==>> Jones, Baker, and Williams
%%   \citeyear{key}         ==>> 1990
%%   \citeyearpar{key}      ==>> (1990)
%%   \citetext{priv. comm.} ==>> (priv. comm.)
%%   \citenum{key}          ==>> 11 [non-superscripted]
%% Note: full author lists depends on whether the bib style supports them;
%%       if not, the abbreviated list is printed even when full requested.
%%
%% For names like della Robbia at the start of a sentence, use
%%   \Citet{dRob98}         ==>> Della Robbia (1998)
%%   \Citep{dRob98}         ==>> (Della Robbia, 1998)
%%   \Citeauthor{dRob98}    ==>> Della Robbia


%% References with bibTeX database:

\bibliographystyle{elsarticle-num-names}
\bibliography{example-bib}

\end{document}

