%%
%% Copyright 2007, 2008, 2009 Elsevier Ltd
%%
%% This file is part of the 'Elsarticle Bundle'.
%% ---------------------------------------------
%%
%% It may be distributed under the conditions of the LaTeX Project Public
%% License, either version 1.2 of this license or (at your option) any
%% later version.  The latest version of this license is in
%%    http://www.latex-project.org/lppl.txt
%% and version 1.2 or later is part of all distributions of LaTeX
%% version 1999/12/01 or later.
%%
%% The list of all files belonging to the 'Elsarticle Bundle' is
%% given in the file `manifest.txt'.
%%

%% Template article for Elsevier's document class `elsarticle'
%% with harvard style bibliographic references
%% SP 2008/03/01
%%
%%
%%
%% $Id: elsarticle-template-harv.tex 4 2009-10-24 08:22:58Z rishi $
%%
%%
\documentclass[final,authoryear,11pt,times]{elsarticle}

%% Use the option review to obtain double line spacing
%% \documentclass[authoryear,preprint,review,12pt]{elsarticle}

%% Use the options 1p,twocolumn; 3p; 3p,twocolumn; 5p; or 5p,twocolumn
%% for a journal layout:
%% \documentclass[final,authoryear,1p,times]{elsarticle}
%% \documentclass[final,authoryear,1p,times,twocolumn]{elsarticle}
%% \documentclass[final,authoryear,3p,times]{elsarticle}
%% \documentclass[final,authoryear,3p,times,twocolumn]{elsarticle}

%% \documentclass[final,authoryear,5p,times,twocolumn]{elsarticle}

%% if you use PostScript figures in your article
%% use the graphics package for simple commands
%% \usepackage{graphics}
%% or use the graphicx package for more complicated commands
%% \usepackage{graphicx}
%% or use the epsfig package if you prefer to use the old commands
%% \usepackage{epsfig}

%% The amssymb package provides various useful mathematical symbols
\usepackage{amssymb}

\usepackage[margin=1.25in]{geometry}


\usepackage{setspace}
\onehalfspacing

%% The amsthm package provides extended theorem environments
%% \usepackage{amsthm}

%% The lineno packages adds line numbers. Start line numbering with
%% \begin{linenumbers}, end it with \end{linenumbers}. Or switch it on
%% for the whole article with \linenumbers after \end{frontmatter}.
%% \usepackage{lineno}

%% natbib.sty is loaded by default. However, natbib options can be
%% provided with \biboptions{...} command. Following options are
%% valid:

%%   round  -  round parentheses are used (default)
%%   square -  square brackets are used   [option]
%%   curly  -  curly braces are used      {option}
%%   angle  -  angle brackets are used    <option>
%%   semicolon  -  multiple citations separated by semi-colon (default)
%%   colon  - same as semicolon, an earlier confusion
%%   comma  -  separated by comma
%%   authoryear - selects author-year citations (default)
%%   numbers-  selects numerical citations
%%   super  -  numerical citations as superscripts
%%   sort   -  sorts multiple citations according to order in ref. list
%%   sort&compress   -  like sort, but also compresses numerical citations
%%   compress - compresses without sorting
%%   longnamesfirst  -  makes first citation full author list
%%
%% \biboptions{longnamesfirst,comma}

% \biboptions{}

\journal{cs280r - Final Project Report}

\begin{document}

\begin{frontmatter}

%% Title, authors and addresses

%% use the tnoteref command within \title for footnotes;
%% use the tnotetext command for the associated footnote;
%% use the fnref command within \author or \address for footnotes;
%% use the fntext command for the associated footnote;
%% use the corref command within \author for corresponding author footnotes;
%% use the cortext command for the associated footnote;
%% use the ead command for the email address,
%% and the form \ead[url] for the home page:
%%
%% \title{Title\tnoteref{label1}}
%% \tnotetext[label1]{}
%% \author{Name\corref{cor1}\fnref{label2}}
%% \ead{email address}
%% \ead[url]{home page}
%% \fntext[label2]{}
%% \cortext[cor1]{}
%% \address{Address\fnref{label3}}
%% \fntext[label3]{}

\title{CS280r Final Project Report \\ $A \rho \mu o \nu \acute{\iota} \alpha$ (Harmonia): A System for Collaborative Music Composition}

%% use optional labels to link authors explicitly to addresses:
%% \author[label1,label2]{<author name>}
%% \address[label1]{<address>}
%% \address[label2]{<address>}

\author{Mark Goldstein, David Wihl}
\address{\{markgoldstein,davidwihl\}@g.harvard.edu}

\begin{abstract}
%% Text of abstract


NOTE: WE SHOULD USE THAT FEATURE THAT LETS YOU SAY "SEE SECTION ANALYSIS ", and it automatically adds in the right section numbers!\\

Increasing productivity of music composition has many positive benefits. Listeners
would appreciate individually tailored music to their emotional needs and context.
Composers would be facilitated by greater and more diverse cooperation yielding more
innovative music. Composition agents could assist in the generation of repetitive or 
experimental musical forms. Therapists can use music as part of a treatment plan 
for autism and
many other disorders. The system we propose attempts to address these myriad needs 
by offering two key innovations: a SharedPlan with collaborative versioning to mediate the workflow of a composition, an algorithmic evaluation of a composition against the intention of the
SharedPlan to provide guidance to both human and agent composers.

\end{abstract}
\end{frontmatter}

% \linenumbers

%% main text
\section{Introduction}
\label{sec:introduction}
TODO: [David]  Should contain an overview of the problem to be addressed, the approach taken to   
  address that problem, and the results of that approach. Should provide the reader with a  
  road map for how your argument will be developed in the other sections of the paper.


RESEARCH TODO
\begin{itemize}
\item Mark: talk to composers and incorporate their feedback and UI suggestions
\item David: speak to David Greenberg to incorporate feedback and UI
\end{itemize}

%
%\section{Body of the Paper}
%
%\begin{itemize}
%	\item {\bf  Experimental Design.} A description of the experiment that was run; enough detail should be provided   
%that the reader could reasonably duplicate the experiment. Results should not be 
%reported in this section.
%	\item {\bf Results.} A report of the results of the experiments, and their significance.
%\end{itemize}


%\subsection{Citations}

%Here are two examples of how to cite a paper properly:
%\begin{itemize}
%	\item \citet{bernstein2000complexity} shows that ... 
%	\item Prior work has shown that ... \citep{bernstein2000complexity}.
%\end{itemize}

%%  \citet{key}  ==>>  Jones et al. (1990)
%%  \citep{key}  ==>>  (Jones et al., 1990)


\section{Related Work}
%TODO Discussion of previous important, similar work in the area with comparison to the particular approach taken and results of the paper. Avoid simply providing a laundry list of other work that is somehow related to the subject of the paper. This section should contain brief, in depth discussions of the work most similar to your project, i.e., to research that takes an approach to the problem or produces results with which your project should be compared. As is always the case with written work, throughout the paper you should have citations to work that you draw on. For example, if you have adapted a system, include a citation to the system when you first mention it; if you are extending a formalization, include a citation to the original on first mention. If you are unclear about whether a simple citation suffices or an extended discussion is needed in the Related Work section, look at the papers read for class this semester for models. If you are still unsure, check with the teaching staff.

Our work builds on several areas of research related to music, computer science, and creative cognition. First we discuss related work in collaborative ideation, both in general and specifically in music. Next we discuss intelligent 
music systems that facilitate human composition and improvisation, and related work in Music Information Retrieval (MIR). Finally we describe previous work that applies information theory to the analysis of musical structure.

\subsection{Shared Plans}
TODO: [David]

\subsection{Collaborative Ideation}

Collaborative Ideation (CI) seeks to improve the productivity of individuals and groups in generating ideas through collaboration. The people involved are interested in creating related objects (e.g., we all want to brainstorm solutions to social problems) and seek either feedback or examples of others' work to enhance their individual process. Collaboration is centered around a shared workspace, physical or virtual, that allows for communication and sharing of ideas. The ideas produced may be for individual use, or ideators may work on shared artifacts such as an essay or piece of art. The dynamics of collaboration may be real-time or not, though increasingly, today's settings are real-time and virtual. A simple CI setting is one where each ideator brainstorms solutions to a problem common to all participants, and each participant can see all other's ideas. While such approaches have been used naturally in human culture for millennia, the design of intelligent computer systems today aims to facilitate these activities to allow for increased creativity and productivity. 

IdeaHound [Siangliulue 2016] addresses at-scale collaboration in this setting. Siangliulue's work claims that only a small subset of the idea pool may be relevant and inspiring to a single ideator, that it is overwhelming for each ideator to view all participant's ideas. The system creates a semantic map of all generated ideas that allows each ideator to easily view their work in the context of the entire solution space. The map is automatically generated. Each user is prompted to interact with a personal ``whiteboard`` where they can cluster their ideas and separate them by semantic distance, and the global map is computed from the collection of whiteboards. This approach bypasses the need for external workers to power semantic analysis of ideas.  Using this map, IdeaHound recommends diverse suggestions to each ideator, eliminating the cognitive load of idea search. Our work is largely influenced by IdeaHound, but several challenges specific to collaborative music composition require new interventions. First, the generated objects are structured rather than unordered collections of ideas. Second, ideators need to build over each other's ideas rather than only seek inspiration.

CI has surfaced in a setting closer to ours, in the space of online blogs and services designed to share visual art and music. Ideas range from small, unfinished efforts seeking directions to finished pieces seeking critique. Artists improve upon their ideas using the large-scale feedback. SoundCloud is an example of a hybrid music streaming service and CI platform. Though much of the hosted music is presented in finished form, people also post incomplete projects. Artists sometimes share ``stems" to their music, which are individual sound files that feature isolated instrumental tracks, with the intention that others seeking inspiration remix their pieces into new work. A newer platform, Blend, makes the sharing of source files explicit. By default, artists share their works in progress in the format of music production software source files, which allows others to quickly pick up on their work and take it in new directions. This setting is closer to our area of application and supports building on one another's ideas.  What changes when several ideators intend to create a single shared piece? With SoundCloud and Blend, one may take another's piece in a totally different direction. In our work, a collaborative composition has a goal associated with it through the duration of its existence. It is up to the composers and the system to keep a piece of music close to its shared plan.

\subsection{Computer Facilitated Composition and Improvisation}

Computer agents with the ability to facilitate and take part in music composition and improvisation are of great interest to music theorists and artificial intelligence researchers. These systems have in common a requirement to ``understand" music at multiple levels, including low-level acoustic signal, mid-level theoretical constructs such as harmony and rhythm, and high-level level aspects such as mood, genre, and style. For example, music recommendation system such as Spotify seek to analyze music and extract a measure of relevance for a function such as "study music". These issues constitute the research area of Music Information Retrieval.

In systems that create music, the interest is to assist human composers, rather than replace them. Perhaps a composer has good ``seeds" ideas, but the system may recommend variations of ideas, or re-orderings of ideas, to make them more conveying. Such system knowledge often comes from large-scale corpus analysis that mines patterns common to a collection of music. ChordRipple [Huang 2016] is a recent system that takes as input a progression of musical chords from a composer, and suggests substitutions of intermediate chords that preserve the original semantics of the input while serving to replace conventional choices with more interesting ones. If the composer agrees to make one of the recommended changes, the system assists the composer in interpolating between original and substituted material before and after the initial substitution, resulting in further mixing of human and system generated music.

While our current work seeks primarily to assist teams of human composers to enrich and organize their work, we intend to design the system such that intelligent computer agent composers may be further in the loop. The Google Brian team has recently launched the Magenta Project for exploring machine intelligence in music. Magenta is an integrated environment of software tools and music-related datasets. Recently, Magenta released AI Duet, a computer system that reacts to human improvised gestures. Improvisation is an important part of composition. Even in steps where a human is composing, it may be beneficial to have an agent for the human to go back-and-forth on ideas with, much like two friends would iteratively vary and refine their ideas. In settings where a piece is defined by a specific enough set of guidelines, such as in a therapy use case where a listener may need music at a certain tempo and with a simple beat [See section INSERT SECTION] , powerful information retrieval systems make effective machine composition agents possible. Human composers may be placed at later steps of collaboration to ensure that the piece meets requirements in a humanly perceptible way.

\subsection{Information Theory and Music Analysis}

Our systems relies on the ability to model musical structure in a way that supports automated feedback for collaborating composers, where feedback is in the form of suggested rearrangements of musical ideas that help a composition reach a mutually specified structural goal. In this direction, there has been a rich body of work in automated analysis of musical structure from the 1950's to present. A prominent direction is to model musical form by way of listener perception and the expected dynamics of their attention and surprise. Understanding musical structure and its impact on listeners is a cogent goal to music theorists, cognitive scientists, and machine learning researchers. Many of these approaches have drawn on probability theory and information theory. A survey of approaches historical to contemporary can be found in the Con Espressione Manifesto [Widmer 2016], which is a strong position paper on the coming devade of research directions for music information reitraval

Our work builds on the the Information Dynamics Approach [Abdallah et al 2012]. Abdallah uses predictive information rate, a entropy/divergence based metric that measures how a listener's mid-piece distribution over future musical events is continually revised as new information is presented, to compute a curve that summarizes aspects of surprise in redundancy in a piece of music. Our work assumes that musical structure can be effectively summarized by this criteria. We assume that pieces from a particular genre/mood are defined by characteristic balances of surprise and redundancy over time, with peaks of information content (communicated by the composer) in genre-specific locations. We leverage this metric as the foundation of our automatic analysis system, which compares a collaborative work-so-far against the characteristic curves for the genre and mood specified by the mutual plan for the piece, and suggests edits to the composers to keep them in line with their goal.

% THESE ARE EXTENSIONS TO HOW WE COULD IMPROVE THE ANALYSIS (DO NOT DELETE)

%In the manifesto, Widner recalls key points made by theorist Cohen in the 1950s, that an information approach for describing music must develop to account for two things crucial to music but not included in information heory (at Cohen's time but still hardly so half a century later) 1) there must be a theory of interactions for multiple streams of musical information, for example for the way in which rhythmic information may make harmonic events more or less certain. Composers do not decide all of music's parameters separately. 2) there must be theory for account for multiple levels of structural hierarchy coming at once from a single stream of musical information. Rhythms constitute a local time feel but also accelerate a piece toward new sections. Recent directions in information theory may provide insight. 

%The first point requires a generalization of mutual information to multiple random variables, which has been difficult to interpret and met with confusion over several coexistent approaches [V.d. Cruys 2011]. Multivariate mutual information is used to describe the extent to which the knowledge of one event can reduce the entropy in several other variables. To our knowledge, the second point has not yet been explored.

%Cohen Quotes
%1. the basic [assumption] is that statistical probability... corresponds to the listener's expectations... the average surprisal value... represents the listener's state of uncertainty?
%2. ?Another... is that one portion of the cultural sign system can be legitimately abstracted from the whole, and that values based on this abstraction will have the same worth as when the portion is a part of the whole.?
%3. ?A further assumption... a sequence of musical events is experienced on only one architectonic level: in melodic analyses, on the level of notes or intervals; in rhythmic analyses, on the level of the pulse pattern... theory will have to take account of the interaction among levels.?
%We need to consider several streams of info. and their interactions: rhythm, pitch, harmony, timbre. Even within one type of information stream, say pitch, we need to consider several hierarchical levels at once: 






\section{System Design}

[perhaps include a workflow image and a MIDI image side-by-side here]

\subsection{Workflow Overview}

TODO [David]

someone creates a shared plan (individual or composers)

information retrieval system gets characteristic metrics using as much of the sharedplan info as it can

composers iteratively work on the piece

Within a particular composer/system interaction, automated analysis is run on the piece as it is.
suggests some actions to the composer, particularly suggesting to switch two parts of a piece.
Composer can follow suggestions or do their own modifications, or do nothing and finish.

\subsection{GIT}

git + intention, attributing credit, version control not available in existing midi systems

TODO [David]: reference splice.com and blend.io



\subsection{Genre + Mood}

TODO: [Mark]

existing information retrieval systems


\subsection{MIDI and actions to be taken on piece of music}
TODO: [Mark]

Our system represents music in MIDI format. MIDI is a protocol for communicating discrete information about the pitch, duration, and dynamics of individual notes, and provides a way to describe the vertical arrangement of individual notes as chords and their horizontal arrangement over time. Most familiar software for music notation and music production build user interfaces on top of a basic MIDI file editor.

Through the use of additional metadata (next section), composers are able to segment MIDI files that represent a piece of music into separate segments. For example, a piece may be subdivided into several sections. Segmentation may represent intentions about musical form, for example one may segment part of a composition into an exposition and a development section.

During an interaction with the system, a composer is able to change a composition by editing the MIDI in several low-level or high-level ways. Composers can add a new block of material, edit an existing block, remove an existing block, swap the order of two blocks, merge two blocks, or split a block into two.

At each iteration of editing, the system suggests an option that may bring the work in progress closer to what is specified by the SharedPlan. This is primarily in the form of "switch blocks A and B?" (See Section on Automatic Evaluation)

\subsection{Shared Plan + Metadata, inter-composer communication}

TODO: [David]

git + intention + algorithmic eval

How do composers or users express intentions?

What sharing of information occurs between users and composers? Decoupling (be explicit)

failure mode: how to avoid revision wars

\section{User Interface}

TODO: [David]








\section{Automated Analysis of Musical Structure}

TODO: [Mark]

\subsection{Brief information theory background}

Entropy

KL Divergence

Mutual Information

\subsection{Current Design}


\section{Use Cases}

\subsection{Individual User, Individual Composer}

TODO: [David]

Our first use case considers the following scenario: a listener who may be a non-musician would like a new piece of music, perhaps to use for a function such as study music. We consider the case that the listener specifies a new project defined by a mood and genre. At this point, multiple composers could collaborate on the music specification, but we first consider the case that a single composer iterates over the piece with assistance from our system until the requester is happy.


\subsection{Multiple Composers}
TODO: [Mark]

Our second use case considers the case where multiple composers create a music specification together, and then collaboratively compose music that stays on track with the original specification.

TODO: include failure modes

\subsection{Therapist with Agent - Human Composition Team}

TODO: [David]

Our third case considers the situation where a music therapist would like music to use with their patients. These pieces may have a more highly-refined specification than music for casual listening. The specification may follow a treatment plan and may require a specific tempo or special therapeutic timbres (sound qualities).

high volume necessity

given the detailed specification, an artificially intelligent agent may do a large amount of work, which is then checked by a human composer


\section{Discussion}

TODO: [David + Mark]
\subsection{Enhancing or Stifling Creativity}

Notes: evaluation is optional. Can be ignored by committer.

\subsection{Limitations}

TODO: [David]
Collaboration is offline, not real-time

Current music representation is discrete MIDI, not audio. Limits for vocals, ocean sounds

Presume that reliable corpus-based genre and mood classification solutions exist, particularly information retrieval procedures



\section{Conclusion}
 Two Novel Contributions:
 \begin{itemize}
\item Collaborative music composition system 
Intentionality, SharedPlan and Agents
\item Algorithmic evaluation of composition against intention
\end{itemize}

\section {Future work}
TODO: [David]

\begin{itemize}
\item Improved agent composition
\item Intelligent ad hoc composition
\item Facilitator of scalable music composition
\item improved evaluator, possibly RNN based
\end{itemize}

\section{References}
Meyer, L.B., 1956. Emotion and Meaning in Music. Chicago University Press, Chicago, IL.

Narmour, E.. 1992. The Analysis and Cognition of Melodic Complexity: The Implication-Realization Model.

D. Huron. 2006. Sweet Anticipation: Music and the Psychology of Expectation. MIT Press, Cambridge, MA.

Abdallah, Cognitive Music Modelling: An Information Dynamics Approach. 2012 3rd International Workshop on Cognitive Information Processing (CIP)

Widmer, Gerhard. 2016. Getting closer to the essence of music: The Con Espressione manifesto. ACM Transactions on Intelligent Systems and Technology

Engel et al., Neural Audio Synthesis of Musical Notes with WaveNet Autoencoders, 2017

Wiggins, Auditory Expectation: The Information Dynamics of Music Perception and Cognition. 2012 Topics in Cognitive Science

Moles, A.. 1966. Information Theory and Aesthetic Perception. University of Illinois Press, Urbana, IL

Two Multivariate generalizations of Pointwise Mutual Information Tim Van de Cruys, Association for Computational Linguistics 2011

Cohen, Joel E., Information Theory and Music , Behavioral Science, 7:2 (1962:Apr.) p.137

Schillinger, Joseph The mathematical basis of the arts 1948

Pierce, Electronics, waves, and messages. 1956

Pierce, Letter Scientific American 1956

Youngblood, Style as information 1958 Journal of Music Theory

The mathematical theory of communication. Shannon, Claude Elwood 1948. Bell Tel Labs Monograph


%% The Appendices part is started with the command \appendix;
%% appendix sections are then done as normal sections
%% \appendix

%% \section{}
%% \label{}

%% References
%%
%% Following citation commands can be used in the body text:
%%
%%  \citet{key}  ==>>  Jones et al. (1990)
%%  \citep{key}  ==>>  (Jones et al., 1990)
%%
%% Multiple citations as normal:
%% \citep{key1,key2}         ==>> (Jones et al., 1990; Smith, 1989)
%%                            or  (Jones et al., 1990, 1991)
%%                            or  (Jones et al., 1990a,b)
%% \cite{key} is the equivalent of \citet{key} in author-year mode
%%
%% Full author lists may be forced with \citet* or \citep*, e.g.
%%   \citep*{key}            ==>> (Jones, Baker, and Williams, 1990)
%%
%% Optional notes as:
%%   \citep[chap. 2]{key}    ==>> (Jones et al., 1990, chap. 2)
%%   \citep[e.g.,][]{key}    ==>> (e.g., Jones et al., 1990)
%%   \citep[see][pg. 34]{key}==>> (see Jones et al., 1990, pg. 34)
%%  (Note: in standard LaTeX, only one note is allowed, after the ref.
%%   Here, one note is like the standard, two make pre- and post-notes.)
%%
%%   \citealt{key}          ==>> Jones et al. 1990
%%   \citealt*{key}         ==>> Jones, Baker, and Williams 1990
%%   \citealp{key}          ==>> Jones et al., 1990
%%   \citealp*{key}         ==>> Jones, Baker, and Williams, 1990
%%
%% Additional citation possibilities
%%   \citeauthor{key}       ==>> Jones et al.
%%   \citeauthor*{key}      ==>> Jones, Baker, and Williams
%%   \citeyear{key}         ==>> 1990
%%   \citeyearpar{key}      ==>> (1990)
%%   \citetext{priv. comm.} ==>> (priv. comm.)
%%   \citenum{key}          ==>> 11 [non-superscripted]
%% Note: full author lists depends on whether the bib style supports them;
%%       if not, the abbreviated list is printed even when full requested.
%%
%% For names like della Robbia at the start of a sentence, use
%%   \Citet{dRob98}         ==>> Della Robbia (1998)
%%   \Citep{dRob98}         ==>> (Della Robbia, 1998)
%%   \Citeauthor{dRob98}    ==>> Della Robbia


%% References with bibTeX database:

\bibliographystyle{elsarticle-num-names}
\bibliography{example-bib}

\end{document}

